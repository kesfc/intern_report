% Author: Mai
% License: LaTeX Project Public License v1.3c
% 完整编译: XeLaTex -> BibTex -> XeLaTex -> XeLaTex

%%%%%%%%%%%%%%%%%%%%%%%%  文档配置  %%%%%%%%%%%%%%%%%%%%%%%%

\documentclass[report,oneside,UTF8,zihao=-4]{config}

% \def \classificationNumber {A0000} % 分类号
% \def \UDC                  {AA000.0} % UDC
% \def \confidentialLevel    {公开} % 密级
% \def \serialNumber         {0000} % 编号

% 封面图片定义
\def \titlePageImages{
    \includegraphics[width=0.195\textwidth] {DGUT-logo.pdf}\\ % 东莞理工学院校徽
    \vspace{10pt}
    \includegraphics[width=0.425\textwidth] {DGUT-title-CN.pdf}\\ % 东莞理工学院中文校名
    \includegraphics[width=0.425\textwidth] {DGUT-title-EN.pdf}\\ % 东莞理工学院英文校名
}

% 文档信息定义
\def \majorTitle   {模板使用指南} % 大标题
\def \minorTitleCN {\LaTeX{} 论文/报告/作业模板(非官方)使用指南} % 中文标题
\def \minorTitleEN {\LaTeX{} Thesis/Report/Assignment Template (Not Official) User Guide} % 英文标题

% 个人信息定义
\def \titlePageInfoBox{
    % 参数:#1 下划线长度 #2 字号 #3 标题 #4 内容
    \infobox{6.00cm}{0.60cm}{版~~~~本}{v3.5.6}\\
    \infobox{6.00cm}{0.60cm}{学~~~~院}{ABCD}\\
    \infobox{6.00cm}{0.60cm}{专~~~~业}{EFGH}\\
    \infobox{6.00cm}{0.60cm}{班~~~~级}{IJKL}\\
    \infobox{6.00cm}{0.60cm}{姓~~~~名}{Mai}\\
    \infobox{6.00cm}{0.60cm}{学~~~~号}{201912345678}\\
    % \infobox{5cm}{0.60cm}{指导教师}{Mentor}\\
}

\def \currentDate {二〇二三年一月一日} % 日期 \zhtoday获取当前中文日期
     
%%%%%%%%%%%%%%%%%%%%%%%%  文档开始  %%%%%%%%%%%%%%%%%%%%%%%%

\begin{document}

% 封面
\CoverPage
    {empty} % 封面类型:empty、both、left、right
    {0.825cm} % 大标题字号
    {0.700cm} % 中文标题字号
    {0.550cm} % 英文标题字号

%%%%%%%%%%%%%%%%%%%%%  正文前页眉页脚  %%%%%%%%%%%%%%%%%%%%%%

% 页眉(关闭页眉务必将页眉类型设为empty)
\Header
    {common} % 页眉类型:common、publish、empty
    {1pt} % 上分隔线宽度
    {1pt} % 两线距离
    {0.5pt} % 下分割线宽度
    {} % 页眉左自定义内容(文本或图片)
    {\includegraphics[width=0.185\textwidth]{DGUT-title-CN.pdf}} % 页眉中自定义内容(文本或图片)
    {} % 页眉右自定义内容(文本或图片)

%============================================%

% 页脚(关闭页脚务必将页脚类型设为empty) 
\Footer
    {common} % 页脚类型:common、publish、empty
    {0pt} % 上分隔线宽度
    {0pt} % 两线距离
    {0pt} % 下分割线宽度
    {} % 页脚左自定义内容(文本或图片)
    {\thepage} % 页脚中自定义内容(文本或图片)
    {} % 页脚右自定义内容(文本或图片)

%============================================%

% 页数样式
\SetRomanPageNumber % 设置罗马数字页码
% \setArabicPageNumber % 设置阿拉伯数字页码
\ResetCounter{1} % 重置页数

%%%%%%%%%%%%%%%%%%%%%%%%  摘要  %%%%%%%%%%%%%%%%%%%%%%%

\begin{abstractCN}[0.65cm] % 中文摘要,参数:#1 中文摘要标题字号

\LaTeX{}(发音为 “Lah-tech” 或 “Lay-tech” )是由 Leslie Lamport 开发的当今世界上最流行和使用最为广泛的 \TeX{} 宏集。它构筑在 PlainTeX 的基础之上,并加进了很多的功能以使得使用者可以更为方便的利用 \TeX{} 的强大功能。

使用 \LaTeX{} 基本上不需要使用者自己设计命令和宏等,因为 \LaTeX{} 已经替你做好了。因此,即使使用者并不是很了解 \TeX{},也可以在短短的时间内生成高质量的文档。对于生成复杂的数学公式,\LaTeX{} 表现的更为出色。

\LaTeX{} 由 \LaTeX{3} 项目维护,很多使用者对 \LaTeX{} 加入了很多补充扩展,例如为 \LaTeX{} 开发宏包和样式,其中的一些已经包含在很多 \LaTeX{} 软件中,可以在CTAN上获得更多的扩展宏包。

% 中文关键词
\def\keywordsCN{关键词1,关键词2,关键词3,关键词4,关键词5}

\end{abstractCN}

%============================================%

\begin{abstractEN}[0.7cm] % 英文摘要,参数:#1 英文摘要标题字号

\LaTeX{} (pronounced either ``Lah-tech" or ``Lay-tech") is a set of macros for \TeX{} created by Leslie Lamport. Its purpose is to simplify \TeX{} typesetting, especially for documents containing mathematical formulae. Within the typesetting system, its name is formatted as ``\LaTeX{}".

\TeX{} is both a typographical and a logical markup language, and one has to take account of both issues when writing a \TeX{} document. On the other hand, Lamport's aim when creating \LaTeX{} was to split those two aspects. A typesetter can make a template and then the writers can just focus on \LaTeX{} logical markup. They might not know anything about typesetting.

In addition to the commands and options \LaTeX{} offers, many other authors have contributed extensions, called packages or styles, which you can use for your documents. Many of these are bundled with most \TeX{}/\LaTeX{} software distributions; more can be found in the Comprehensive \TeX{} Archive Network (CTAN).

% 英文关键词
\def\keywordsEN{keyword 1,keyword 2,keyword 3,keyword 4,keyword 5}

\end{abstractEN}

%%%%%%%%%%%%%%%%%%%%%%%  启用目录  %%%%%%%%%%%%%%%%%%%%%%%%

% 参数: 
% #1 目录类型:next、current
% #2 目录行距
% #3 目录标题
% #4 当前章节名
\contentPage{next}{1.5}{目~~~~录}{目录}
\contentpageOfFigures{next}{1.5}{图目录}{图目录}
\contentpageOfTables{next}{1.5}{表目录}{表目录}
       
%%%%%%%%%%%%%%%%%%%%%%%  启用水印  %%%%%%%%%%%%%%%%%%%%%%%%

% \imageWatermark % 图片水印
%     {0} % 旋转角度
%     {0.8} % 放缩倍率
%     {0.02} % 透明度 0-1
%     {DGUT-logo.eps} % 图片路径

%%%%%%%%%%%%%%%%%%%%%  正文页眉页脚  %%%%%%%%%%%%%%%%%%%%%%%

% 页眉(关闭页眉务必将页眉类型设为empty)
\Header
    {common} % 页眉类型:common、publish、empty
    {1pt} % 上分隔线宽度
    {1pt} % 两线距离
    {0.5pt} % 下分割线宽度
    {使用指南} % 页眉左自定义内容(文本或图片)
    {} % 页眉中自定义内容(文本或图片)
    {\currentChapterInfo} % 页眉右自定义内容(文本或图片)

%============================================%

% 页脚(关闭页脚务必将页脚类型设为empty) 
\Footer
    {common} % 页脚类型:common、publish、empty
    {0pt} % 上分隔线宽度
    {0pt} % 两线距离
    {0pt} % 下分割线宽度
    {} % 页脚左自定义内容(文本或图片)
    {\thepage} % 页脚中自定义内容(文本或图片)
    {} % 页脚右自定义内容(文本或图片)

%============================================%

% 页数样式
% \SetRomanPageNumber % 设置罗马数字页码
\SetArabicPageNumber % 设置阿拉伯数字页码
\ResetCounter{1} % 重置页数

%%%%%%%%%%%%%%%%%%%%%%%  正文  %%%%%%%%%%%%%%%%%%%%%%%%%%

\chapter{关于本模板}

\section{介绍}

\LaTeX{} 是一个强大的文档排版工具,但其相对较大的学习成本往往让初学者难以上手使用。为了让初学者能轻松使用 \LaTeX{},作者通过对一些成熟的方案进行整合、封装,创作出本模板。

本使用指南演示了模板中的一些典型使用场景,其中包括:

\begin{itemize}
    \item \textbf{排版相关:}封面页、页眉页脚、目录样式、标题样式、图表样式、脚注等。
    \item \textbf{文本相关:}多级标题、列表。
    \item \textbf{图表相关:}单个图片、多个图片。普通表格、复杂表格、跨页表格。
    \item \textbf{引用相关:}多种样式的文献引用。
    \item \textbf{数学相关:}数学符号、公式、定理证明。
    \item \textbf{代码相关:}算法(伪代码)、代码块。
\end{itemize}

作者在开发本模板时大量参考、模仿、学习了许多优秀的第三方模板,参考的内容和参考模板的地址如表 \ref{tab:templates} 所示。

\begin{table}
    \centering
    \renewcommand{\arraystretch}{1.3} % 定义表格行距
    \setlength{\tabcolsep}{3pt} % 定义列间宽度
    \caption{参考模板相关信息}
    \label{tab:templates}
    \begin{threeparttable}[c]
        \begin{tabular}{ccc}
            \toprule[1.5pt]
            \textbf{参考模板}    &  \textbf{参考内容}    & \textbf{模板地址}\\
            \midrule[0.8pt]
            清华大学学位论文模板 & 示例文档、部分模板源码 & \href{https://github.com/tuna/thuthesis}{Github 地址}\\
            华南师范大学本科毕业论文模板 & 部分模板源码 & \href{https://www.overleaf.com/latex/templates/scnu-my-article/jkbbvhnddtsw}{OverLeaf 地址}\\
            上海师范大学研究生毕业论文模板 & 部分模板源码 & \href{https://www.overleaf.com/latex/templates/shnu-thesis/wsykzrksspgn}{OverLeaf 地址}\\
            武汉大学博士论文模板 & 部分模板源码 & \href{https://www.overleaf.com/latex/templates/wu-yi-da-xue-bo-shi-lun-wen-latex-mo-ban/rcdzgvqgkddk}{OverLeaf 地址}\\
            南开大学毕业论文模板 & 部分模板源码 & \href{https://github.com/Tr0py/NKU-thesis-template-2020}{Github 地址}\\
            复旦大学博士学位论文模板 & 部分模板源码 & \href{https://cn.overleaf.com/latex/templates/fduthesis-latex-thesis-template-for-fudan-university/svtdhhstkmkt}{OverLeaf 地址}\\
            厦门大学博士学位论文模板 & 部分模板源码 & \href{https://www.overleaf.com/latex/templates/jing-ji-xue-yuan-yu-wang-ya-nan-jing-ji-yan-jiu-yuan-lun-wen-mo-ban/wgsmpvqhhvzf}{OverLeaf 地址}\\
            \bottomrule[1.5pt]
        \end{tabular}
    \end{threeparttable}
\end{table}

\section{协议}

本模板采用 LPPL v1.3c\footnote{\url{https://www.latex-project.org/lppl/lppl-1-3c/}}或其之后的版本进行许可,请在遵循许可的前提下使用本模板。

\section{模板文件结构}

本模板的文件目录结构如表 \ref{tab:文件目录} 所示。

\begin{table}
  \centering
  \caption{模板文件目录结构}
  \label{tab:文件目录}
  % \renewcommand\arraystretch{1.2} % 定义表格行距
  \setlength{\tabcolsep}{30pt} % 定义列间宽度
  \begin{tabular}{ll}
    \toprule[1.5pt]
    \textbf{文件名}           & \textbf{说明}\\
    \midrule[0.8pt]
    fonts                    & 字体文件夹\\
    images                   & 图片文件夹\\
    config.cls               & 模板文件\\
    \textbf{guide.tex}       & \textbf{使用指南文本文件}\\
    License                  & 软件许可证文件\\
    \textbf{main.tex}        & \textbf{空白文本文件}\\
    references.bib           & 参考文献表样式文件\\
    \bottomrule[1.5pt]
  \end{tabular}
\end{table}

其中,.cls 文件中定义了宏包、指令等,可以大致理解为模板的配置文件、源代码文件;.tex 文件用于编写主要的文本内容;.bib 文件是 \hologo{BibTeX} 所使用的文件,用于存放引用文献的相关信息。

\file{fonts} 文件夹用于存放模板中用到的字体。\file{images} 文件夹用于存放模板中用到的图片,\file{images/logos} 文件夹中存放的图片被用于封面页和水印。

\section{模板使用方式}

因本模板的所有编写都在 OverLeaf\footnote{OverLeaf 是一个强大的线上编辑 \LaTeX{} 文档工具,提供文档预览、线上存储、多人协作、版本管理、模板分享等功能。} 上完成,所以本文仅介绍模板在 OverLeaf 上的使用方法。

要在 OverLeaf 上使用本模板,只需要将模板的 rar 压缩文件上传至 OverLeaf 的个人项目库中并打开即可。使用模板时,务必保证使用的编译器为 \hologo{XeLaTeX}。

熟悉本模板以及 \LaTeX{} 的基本操作后,建议将主文档文件切换为 \file{main.tex} 直接使用。

\chapter{文本相关}

本章节介绍使用 \LaTeX{} 完成简单的文本操作,其中包括分级标题、段落、列表(部分功能)的使用演示。文本内容选取自 \LaTeX{} 的维基教科书介绍\footnote{\url{https://en.wikibooks.org/wiki/LaTeX/Introduction}}。

\section{了解 \TeX{}}

\TeX{}(X 或 $\chi$)是由 Donald Knuth 创造的基于低级编程语言的电子排版系统,利用 \TeX{} 能够对文章进行十分精美的排版。\TeX{} 提供了一套功能强大并且十分灵活的排版语言,它多达 900 多条指令,支持 if-else 判断语句和运算(运算在文档编译的过程中执行)等功能,并且 \TeX{} 有宏功能,用户可以不断地定义自己适用的新命令来扩展 \TeX{} 系统的功能。许多人利用 \TeX{} 提供的宏定义功能对 \TeX{} 进行了二次开发,其中比较著名的有美国数学学会(AMS)推荐的非常适合于数学家使用的 \hologo{AmSLaTeX} 以及适合于一般文章、报告、书籍的 \LaTeX{} 系统。

\TeX{} 系统是公认的数学公式排得最好的系统。美国数学学会鼓励数学家们使用 \TeX{} 系统向它的期刊投稿。世界上许多一流的出版社如 Kluwer、Addison-Wesley、牛津大学出版社等也利用 \TeX{} 系统出版书籍和期刊。

\TeX{} 的强大之处在于其能够对文档的排版进行非常精细的操作,但也会造成难度的提高和耗时增加。1977 年,随着数字印刷的逐步发展,Donald Knuth 发现了数字印刷的潜力并开始 \TeX{} 排版引擎的编写工作,以改善日益恶化的印刷质量。我们现在用的 \TeX{} 排版系统发布于 1982 年,在 1989 年为了更好地支持多国语言又进行了一些改进。\TeX{} 具有很好的稳定性,能够在许多不同种类的计算机上运行,几乎不出现错误。

\TeX{} 一词来源于希腊语“τεχνολογία”(“technologìa”),意为“科技”;它的第一个音节“τεχ”与“TEX”相似,因此英文名取作“TEX”。\TeX{} 系统以逐渐收敛到 $\pi$ 的数字作为版本号。

\section{了解 \LaTeX{}}

\LaTeX{}(发音为“Lah-tech”或“Lay-tech”)是由 Leslie Lamport 开发的当今世界上最流行和使用最为广泛的 \TeX{} 宏集。它构筑在 PlainTeX 的基础之上,并加进了很多的功能以使得使用者可以更为方便的利用 \TeX{} 的强大功能。使用 \LaTeX{} 基本上不需要使用者自己设计命令和宏等,因为 \LaTeX{} 已经替你做好了。因此,即使使用者并不是很了解 \TeX{},也可以在短短的时间内生成高质量的文档。对于生成复杂的数学公式,\LaTeX{} 表现的更为出色。\LaTeX{} 由 \LaTeX{3} 项目维护,很多使用者对 \LaTeX{} 加入了很多补充扩展,例如为 \LaTeX{} 开发宏包和样式,其中的一些已经包含在很多 \LaTeX{} 软件中,可以在 CTAN 上获得更多的扩展宏包。

\subsection{\LaTeX{} 的缺点}

% 无序列表
\begin{itemize}
    \item 一般来说是不能在输入文章的同时看到最终的输出效果,但是将文章用 \LaTeX{} 编译之后,是可以在屏幕上预览最终的输出效果的。尽管在预先定义好的版面中可以调节一些参数,设计全新的版面还是很困难的,需要耗费大量的时间。
    \item \LaTeX{} 允许在预先定义好的版面中可以调节一些参数以修改样式,但设计一个全新的版面还是很困难的,需要耗费大量的时间。
    \item 需要掌握一些 \LaTeX{} 的排版命令。
    \item \LaTeX{} 不适合于排版非结构化的、无序的文档。
\end{itemize}

\subsection{\LaTeX{} 的优点}

% 有序列表
\begin{enumerate}
    \item 提供专业级的排版设计,使你的文档看起来如同印刷好的一样。
    \item 可以更方便地排版数学公式。
    \item 用户仅仅需要掌握少数容易理解的,用来说明文档逻辑结构的命令,而无需对实际的页面设计做胡乱的修补。
    \item 可以很容易地生成脚注、索引、目录和参考文献等复杂的结构。
    \item 有大量免费的可添加宏包,协助你完成许多基本的 \LaTeX{} 未直接支持的排版任务。例如,支持在文档中插入PostScript图形的宏包和排版符合各类标准的参考文献的宏包等。
    \item \LaTeX{} 支持DVI和PDF格式的输出,如配合其他软件可以很容易地生成 PostScript、PNG、JPG 等格式的输出。
    \item \TeX{} 作为 \LaTeX{} 的格式化引擎,是免费软件并且具有极高的可移植性,因此它几乎可以在任何硬件平台上运行。
\end{enumerate}

\chapter{图表相关}

\section{图片}

\subsection{单个图片}

图片通常在 figure 环境中使用 includegraphics 插入,如图 \ref{fig:example1} 的源代码。建议使用矢量图片(PDF)。照片建议使用 JPG 格式。其他的栅格图建议使用无损的 PNG 格式。

\begin{figure}[H] % 图片位置固定
    \centering % 图片居中
    \includegraphics[width=0.8\linewidth]{example-image} % 图片路径
    \caption*{图片的说明文字。} % 图片说明文字
    \caption{图片的标题} % 图片标题
    \label{fig:example1} % 图片标签
\end{figure}

图片可以通过 width 参数来设置宽度,设置宽度后长度会等比例放缩。一般 width 参数会搭配 \cs{linewidth} 使用,以实现按照当前页面宽度进行等比例放缩。使用的效果如图 \ref{fig:example2} 所示。

\begin{figure}[H]
    \centering
    \includegraphics[width=0.5\linewidth]{example-image}
    \caption{一张缩小到当前页宽的 0.5 倍的图片}
    \label{fig:example2}
\end{figure}

% \clearpage

% 使用 \pkg{tikz} 宏包可以直接画图,该宏包的学习成本和使用难度极高,不建议初学者使用。以下图片是 \pkg{tikz} 的一些使用示例。

% \begin{figure}[H]
%     \centering
%     \input{images/Tikz/玻色爱因斯坦分布示意图}
%     \caption{玻色爱因斯坦分布示意图} 
%     \label{fig:Bose Einstein distribution}
% \end{figure}

% \begin{figure}[H]
%     \centering
%     \input{images/Tikz/散点图}
%     \caption{散点图} 
%     \label{fig:scatter diagram}
% \end{figure}

% \clearpage

\subsection{多个图片}

本模板使用 \pkg{subcaption} 宏包来处理分图。使用的效果如图 \ref{fig:multi-image-01} 、图 \ref{fig:multi-image-02} 所示。

\begin{figure}[H]
    \centering
    \subcaptionbox{\label{fig:subfig-a-1}}{\includegraphics[width=0.25\linewidth]{example-image-a.pdf}}
    \subcaptionbox{\label{fig:subfig-b-1}}{\includegraphics[width=0.25\linewidth]{example-image-b.pdf}}
    \caption{多个独立标题分图示例 01}
    \label{fig:multi-image-01}
\end{figure}

\begin{figure}[H]
    \centering
    \subcaptionbox{分图 A1\label{fig:subfig-a-2-01}}{\includegraphics[width=0.15\linewidth]{example-image-a.pdf}}
    \subcaptionbox{分图 B1\label{fig:subfig-b-2-01}}{\includegraphics[width=0.15\linewidth]{example-image-b.pdf}}
    \subcaptionbox{分图 A2\label{fig:subfig-a-2-02}}{\includegraphics[width=0.15\linewidth]{example-image-a.pdf}}
    \subcaptionbox{分图 B2\label{fig:subfig-b-2-02}}{\includegraphics[width=0.15\linewidth]{example-image-b.pdf}}
    \caption{多个独立标题分图示例 02}
    \label{fig:multi-image-02}
\end{figure}

多个分图可以以多行的形式展示,使用的效果如图 \ref{fig:multi-image-03} 所示。

\begin{figure}[H]
    \centering
    \subcaptionbox{分图 A1\label{fig:subfig-a1-3}}{\includegraphics[width=0.25\linewidth]{example-image-a.pdf}}
    \subcaptionbox{分图 A2\label{fig:subfig-a2-3}}{\includegraphics[width=0.25\linewidth]{example-image-a.pdf}}
    \\
    \subcaptionbox{分图 B1\label{fig:subfig-b1-3}}{\includegraphics[width=0.25\linewidth]{example-image-b.pdf}}
    \subcaptionbox{分图 B2\label{fig:subfig-b2-3}}{\includegraphics[width=0.25\linewidth]{example-image-b.pdf}}
    \caption{多个独立标题分图示例 03}
    \label{fig:multi-image-03}
\end{figure}

两个图左右并排放置, 共用一个标题,使用的效果如图 \ref{fig:multi-image-04} 所示。

\begin{figure}[H]
\centering
    \includegraphics[width=0.25\linewidth]{example-image-a.pdf}
    \includegraphics[width=0.25\linewidth]{example-image-b.pdf}
    \caption{多个共用标题分图示例}
    \label{fig:multi-image-04}
\end{figure}

minipage 也可以实现排版并排插图, minipage 可以划分出虚拟的区块,每个区块中可以进行独立排版,使用的效果如图 \ref{fig:minipage-1} 和  \ref{fig:minipage-2} 所示。

\begin{figure}[H]
    \centering
    
    \begin{minipage}[H]{0.37\linewidth} % 第一个minipage
        \centering
        \includegraphics[width=0.49\linewidth]{example-image-a.pdf}
        \includegraphics[width=0.49\linewidth]{example-image-b.pdf}
        \\
        \includegraphics[width=0.49\linewidth]{example-image-a.pdf}
        \includegraphics[width=0.49\linewidth]{example-image-b.pdf}
        \caption{图 A1、图 B1、图 A2、图 B2}
        \label{fig:minipage-1}
    \end{minipage}
    \begin{minipage}[H]{0.37\linewidth} % 第二个minipage
        \centering
        \includegraphics[width=\linewidth]{example-image-c.pdf}
        \caption{图 C}
        \label{fig:minipage-2}
    \end{minipage}
    
\end{figure}

\section{表格}

\subsection{普通表格}

在 \LaTeX{} 中,表格的编辑相对较为复杂,推荐使用 Table Generator\footnote{\url{https://www.tablesgenerator.com/}} 生成表格。

\begin{table}[H] % 表格位置固定
    \centering % 表格整体居中
    \caption{表格示例} % 表格表题
    \label{tab:example} % 表格标签
    \renewcommand\arraystretch{2.5} % 定义表格行距
    \setlength{\tabcolsep}{30pt} % 定义列间宽度
    \begin{tabular}{|c|c|c|}  % 表格列样式定义
        \hline % 行线
        2 & 9 & 4 \\ % 表内容和换行
        \hline % 行线
        7 & 5 & 3 \\ % 表内容和换行
        \hline % 行线
        6 & 1 & 8 \\ % 表内容和换行
        \hline % 行线
    \end{tabular}
\end{table}

为了使期刊论文的表格结构简洁和满足国际通用规则,通常需要使用三线表。三线表是传统网格线表经过简化改造而来的,取消了斜线、竖线和行线(横向分割线),表 \ref{tab:three-line} 是一个三线表的示例。

\begin{table}[H] % 表格位置固定
    \centering % 表格整体居中
    \caption{三线表示例} % 表格表题
    \label{tab:three-line} % 表格标签
    \renewcommand\arraystretch{1.2} % 定义表格行距
    \setlength{\tabcolsep}{10pt} % 定义列间宽度
    \begin{tabular}{lll} % 表格列样式定义
        \toprule[1.5pt] % 顶线
        \textbf{表头1} & \textbf{表头2} & \textbf{表头3} \\ % 表头
        \midrule[0.8pt] % 栏目线
            test & test & test \\ % 表体
            test & test & test \\ % 表体
            test & test & test \\ % 表体
            test & test & test \\ % 表体
            test & test & test \\ % 表体
        \bottomrule[1.5pt] % 底线
    \end{tabular}
\end{table}

表格如果有附注,尤其是需要在表格中进行标注时,可以使用 \pkg{threeparttable} 宏包。使用的效果如表 \ref{tab:three-line-with-note} 所示。

\begin{table}[H]
    \centering
    \caption{带附注的三线表示例}
    \label{tab:three-line-with-note}
    \renewcommand\arraystretch{1.3} % 定义表格行距
    \setlength{\tabcolsep}{20pt} % 定义列间宽度
    \begin{threeparttable}[c]
        \begin{tabular}{ll}
            \toprule[1.5pt]
            \textbf{文件名}           & \textbf{说明}\\
            \midrule[0.8pt]
            config.cls               & 模板文件\\
            guide.tex\tnote{a}       & 使用指南文本文件\\
            main.tex                 & 空白文本文件\\
            references.bib           & 参考文献表样式文件\\
            \bottomrule[1.5pt]
        \end{tabular}
        \begin{tablenotes}
            \item [a] 可以通过 \hologo{XeLaTeX} 编译生成模板的使用说明文档。
        \end{tablenotes}
    \end{threeparttable}
\end{table}

\subsection{复杂表格}

\LaTeX{} 支持绘制复杂的表格,如表 \ref{tab:complesTable1}、表 \ref{tab:complesTable2} 、表 \ref{tab:complesTable3} 所示。

\begin{table}[H]
    \centering
    \caption{数值误差与收敛速率示例}
    \renewcommand\arraystretch{1} % 定义表格行距
    \setlength{\tabcolsep}{8pt} % 定义列间宽度
    \label{tab:complesTable1}
    \begin{tabular}{|c|c|cc|cc|cc|}
        \Xhline{2\arrayrulewidth}
        degree &  step-size~$h$  & $L^2$-errors  &  order  & $H^1$-errors & order & $L^\infty$-errors  &  order \\
        \hline
           &  1/128    & 9.18E-06    &2.02    & 7.70E-03  &1.01  & 6.46E-07    &2.02 \\
        1  &  1/256    & 2.29E-06    &2.01    & 1.92E-03  &1.00  & 1.61E-07    &2.01 \\
           &  1/512    & 5.70E-07    &2.00    & 9.56E-04  &1.00  & 4.01E-08    &2.00 \\
        \hline % \cline{1-8}
           &  1/128    & 1.39E-08    &3.01    & 1.15E-05  &2.01  & 3.48E-12   &4.02  \\
        2  &  1/256    & 1.73E-09    &3.01    & 2.88E-06  &2.01  & 3.27E-13   &3.94  \\
           &  1/512    & 2.17E-10    &3.00    & 7.24E-06  &2.00  & 6.66E-13   &1.55  \\
        \hline % \cline{1-8}
           &  1/32     & 2.28E-09    &4.05    & 6.92E-07  &3.04  & 1.45E-15   &8.21  \\
        3  &  1/64     & 1.42E-10    &4.03    & 8.65E-08  &3.02  & 2.06E-14   &3.85  \\
           &  1/128    & 8.91E-12    &4.01    & 1.08E-08  &3.01  & 3.86E-14   &0.91  \\
        \Xhline{2\arrayrulewidth}
    \end{tabular}
\end{table}

\begin{table}[H]  
    \centering
    \caption{Compare with other approachs}  
    \label{tab:complesTable2}
    \renewcommand\arraystretch{1.2} % 定义表格行距
    \setlength{\tabcolsep}{10pt} % 定义列间宽度
    \begin{tabular}{|c|c|c|c|c|c|c|}
        \hline
        \multirow{2}*{Model} & \multicolumn{3}{c|}{trigger identification} &  \multicolumn{3}{c|}{Event Extraction} \\ 
        \cline{2-7}
        & P(\%) & R(\%) & F1(\%) & P(\%) & R(\%) & F1(\%) \\
        \hline 
        Baseline1 & 76.84 & 76.84 & 76.84 & 76.84 & 76.84 & 76.84 \\
        \cdashline{2-7}[1pt/2pt]
        Baseline2  & 76.84 & 76.84 & 76.84 & 76.84 & 76.84 & 76.84 \\
        Baseline3  & 76.84 & 76.84 & 76.84 & 76.84 & 76.84 & 76.84 \\
        \cdashline{2-7}[1pt/2pt]
        {\bf Our approach}  & {\bf 76.84} & {\bf 76.84} & {\bf 76.84} & {\bf 76.84} & {\bf 76.84} & {\bf 76.84} \\
        \hline
    \end{tabular}
\end{table}

\begin{table}[H]
    \centering
    \caption{朴素随机加权自助法的经验水平}\label{RW_size}
    \label{tab:complesTable3}
    \renewcommand\arraystretch{1.0} % 定义表格行距
    \setlength{\tabcolsep}{6pt} % 定义列间宽度
    \scalebox{1}{
        \begin{tabular}{ccccccccccccc}
            \hline
            \multicolumn{3}{c}{}& &\multicolumn{4}{c}{ $\alpha=5\%$} &&  \multicolumn{4}{c}{$\alpha=10\%$}\\
            \cline{5-8} \cline{10-13}
            n&检验&权重$\diagdown m$&&2&3&6&12&&2&3&6&12\\
            \hline
            \multirow{5}{*}{$100$}&$\text{LB}^{\chi^2}_m$&&&5.0&3.8&4.0&4.2&&9.6&9.0&8.8&8.4\\
            &$\text{LB}^{\text{HAC}}_m$&&&1.7&1.0&0.3&0.1&&6.2&3.7&1.7&0.5\\
            \linespread{0.5}
            &\multirow{2}{*}{$\text{LB}^{\text{NRW}}_m$}&Exp(1)&&\textbf{4.8}&\textbf{4.2}&\textbf{4.1 }&\textbf{3.8}&& \textbf{9.8}&\textbf{9.2} &\textbf{10.0} &\textbf{10.2}\\
            &&Bernoulli&&\textbf{4.4}&\textbf{3.4}&\textbf{3.2}&\textbf{2.7}&&\textbf{8.2}&\textbf{9.2}&\textbf{7.9}&\textbf{8.4}\\
            \hline
            \multirow{5}{*}{$500$}&$\text{LB}^{\chi^2}_m$&&&5.0&4.7&4.7&4.8&&9.4&10.2&9.1&9.5\\
            &$\text{LB}^{\text{HAC}}_m$&&&3.0&2.7&1.7&0.9&&8.9&7.3&5.3&3.9\\
            \linespread{0.5}
            &\multirow{2}{*}{$\text{LB}^{\text{NRW}}_m$}&Exp(1)&&\textbf{4.8}&\textbf{4.7}&\textbf{4.3}&\textbf{4.2}&&  \textbf{9.6}&\textbf{10.3}&\textbf{8.6}&\textbf{8.5}\\
            &&Bernoulli&&\textbf{4.9}&\textbf{4.6}&\textbf{4.2}&\textbf{3.5}&&\textbf{9.0}&\textbf{9.6}&\textbf{8.5}&\textbf{8.6}\\
            \hline
        \end{tabular}
    }
\end{table}

如表格过宽,可以使用 \pkg{rotating} 宏包中的 sidewaystable 环境旋转表格。使用效果如表 \ref{tab:complesTable4} 所示。

\begin{sidewaystable}[htbp]
    \centering
    \caption{VAR(2) 模型的误差项是 M1 和 M2 下的经验水平和经验功效(\%)}
    \label{tab:complesTable4}
    % \renewcommand\arraystretch{1.1} % 定义表格行距
    % \setlength{\tabcolsep}{5pt} % 定义列间宽度
    \scalebox{0.70}{
        \begin{tabular}{c c c c c c c c c c c c c c c c c c c c c c c c c c c c c c c c c c c}
            \hline
            \multicolumn{4}{c}{}&\multicolumn{15}{c}{ M1}&&\multicolumn{15}{c}{M2} \\
            \cline{1-19} \cline{21-35}
            \multicolumn{4}{c}{ $n$\ \ \ \ \ \ 检验\ \ \ \ \ \ \ 权重  \ \  \ \ \ \ \ $b_n$}& \multicolumn{3}{c}{ $\delta=0.0$} & &  \multicolumn{3}{c}{ $\delta=0.1$}& &\multicolumn{3}{c}{ $\delta=0.2$}& &\multicolumn{3}{c}{ $\delta=0.4$}&& \multicolumn{3}{c}{ $\delta=0.0$} & &  \multicolumn{3}{c}{ $\delta=0.1$}& &\multicolumn{3}{c}{ $\delta=0.2$}& &\multicolumn{3}{c}{ $\delta=0.4$}\\
            \cline{5-7} \cline{9-11} \cline{13-15} \cline{17-19}  \cline{21-23}  \cline{25-27}  \cline{29-31}  \cline{33-35}
            \multicolumn{4}{c}{ \ \ \ \ \ \ \ \ \ \ \ \ \ \ \ \ \ \ \ \ \ \ \ \ \ \ (核函数)\  \ \ \ \ \ $(p_n)$}&1\%&5\%&10\%&&1\%&5\%&10\%&&1\%&5\%&10\%&&1\%&5\%&10\%&&1\%&5\%&10\%&&1\%&5\%&10\%&&1\%&5\%&10\%&&1\%&5\%&10\%\\
            \cline{1-19} \cline{21-35}
            500&$\widehat{\text{CvM}}_{n}^*$&\multirow{5}{*}{\text{Bernoulli}}&1& 1.1&5.3&11.5&&5.0&15.8&25.9&&31.4&60.0&71.2&&99.0&99.6& 99.8  &&1.6&6.5&12.3&&4.7&11.0&19.9&&15.1&32.3&40.4&&56.9&75.1&82.2\\
            \scriptsize &&&3&1.3&5.9&12.1&&5.1&16.3&26.3&&32.2&60.1&72.5&&99.0&99.9&99.9&&1.1&6.8&12.9&&4.2&13.1&21.0&&16.0&33.2&44.4&&53.8& 74.1& 83.1\\
            \scriptsize &&&5&1.5&6.2&12.4&&4.8&15.0&26.3&&32.7&59.9&73.2&&99.0&99.7&99.9&&1.4&7.3&13.6&&3.5&12.5&21.1&&16.1&32.7&44.1&&53.5& 72.4& 82.3\\
            \scriptsize &&&6&1.4&6.3&12.2&&5.4&16.4&27.6&&34.3&60.1&73.4&&99.4&99.3&99.9&&1.7&7.2&12.6&&4.1&12.3&21.5&&14.9&32.9&43.2&&53.0& 73.6& 83.5\\
            \scriptsize &&&9&1.4&6.4&12.3&&5.3&16.1&27.3&&33.3&59.1&72.4&&99.3&99.9&99.9&&1.4&6.3&13.6&&3.8&12.4&21.6&&14.1&32.5&42.5&&52.4& 73.3& 82.0\\
            \scriptsize &&&11&1.5&6.1&12.8&&4.7&15.7&27.3&&33.5&60.6&72.6&&99.3&99.9&99.9&&1.7&6.3&13.4&&3.5&13.1&22.3&&14.2&32.1&43.1&&52.2& 71.6& 82.7\\
            \cline{4-19} \cline{21-35}
            \scriptsize & &\multirow{5}{*}{\text{\text{Exp(1)}}}&1&1.0&4.9&11.3&&3.7&15.4&26.2&&29.1&57.3&72.0&&98.9&99.9&99.9 &&0.4&4.9&12.4&&1.6&9.6&19.5&&1.9&9.7&20.1&&37.6&71.0&82.7\\
            \scriptsize &&&3&0.8&4.7&11.1&&2.7&13.8&24.6&&24.4&55.7&71.1&&98.5&99.9&99.9&&0.6&4.5& 13.1&&1.2&10.1&21.0&& 5.2&26.2&43.3&&23.6&67.5&83.1\\
            \scriptsize &&&5&0.7&4.1&11.1&&2.1&12.5&23.5&&20.0&52.5&69.9&&97.8&99.6&99.9&&0.3&4.8& 12.7&&0.9&8.6&20.8&&3.7&24.2&42.5&&23.3&65.0&83.6\\
            \scriptsize &&&6&0.5&3.9&9.4&&2.2&11.5&23.4&&19.7&51.8&69.0&&97.6&99.7&99.9&&0.4&4.1& 11.7&&0.8&8.5&20.1&&4.3&25.0&42.8&&21.2&64.6&83.7\\
            \scriptsize &&&9&0.3&3.7&9.7&& 2.1&10.0&22.7&&16.4&49.9&68.4&&96.5&99.7&99.9&&0.2&3.5& 11.8&&0.8&8.5&20.1&& 3.7&24.5&42.4&&20.4&64.2&83.4\\
            \scriptsize &&&11&0.8&3.6&9.4&&1.8&10.1&20.9&&15.8&48.7&67.0&& 96.2&99.8&99.9&&0.1&3.8& 11.7&&0.4&8.2&20.9&&3.1&23.7&42.5&& 19.9&62.8&83.7\\
            \cline{2-19} \cline{21-35}
            \scriptsize &$\text{T}_{n}$&\multirow{3}{*}{\text{Daniell}}&6& 0.0&4.6&12.2&&0.9&4.3&9.4&&7.2&15.4&21.8&&89.3&94.4&96.1 &&7.5& 13.6& 21.9&&1.1&4.8&9.9&&1.9&9.6&15.3&&35.7&67.0&75.8\\
            \scriptsize&&&13&0.4&3.6&10.8&&0.8&4.3&8.3&&5.1&10.4&16.3&&74.7&85.5&89.9&&4.6& 14.5& 22.0&&1.0&5.7&10.1&&2.0&11.0&19.8&&36.8&67.1 &76.4\\
            \scriptsize &&&23&1.2&6.2&12.2&&0.9&4.1&9.2&&3.8&9.5&14.6&&61.5&76.4&82.8&&6.6& 15.0& 22.7&&0.2&3.6&7.4&&1.2&8.3&15.8&&16.3&54.6& 66.6\\
            \cline{4-19} \cline{21-35}
            \scriptsize & &\multirow{3}{*}{\text{Parzen}}&6&0.1&1.7&5.5&&1.9&5.5&9.7&&19.9&31.0&37.5&&98.2&99.0&99.0 &&9.6&14.8&20.0&&1.7&6.8&13.0&&5.7&18.1&25.5&&60.5&82.2&89.0\\
            \scriptsize &&&13&0.3&4.7& 10.5&&1.0&4.6&10.1&&11.5& 21.0&27.8&&95.4&97.3&98.2&&7.0&14.7&22.6&&1.3&6.0&11.5&& 2.9& 16.1& 24.3&& 46.0&76.4& 84.2\\
            \scriptsize &&&23&0.3&4.6& 11.0&&1.0&3.8&8.7&&8.3& 16.4& 22.4&&90.5&94.8& 96.2&&6.5&14.1&23.3&&0.9&5.5&10.9&&3.0&12.9&21.1&&40.3&70.2 &79.6\\
            \cline{1-19} \cline{21-35}
             1000&$\widehat{\text{CvM}}_{n}^*$&\multirow{5}{*}{\text{Bernoulli}}&1&1.2&5.4&10.6&&15.3&32.4&45.2 &&75.6&91.3&95.4&&100.0&100.0&100.0  &&0.6&4.2&10.1&&6.2&16.7&25.0&&27.3&47.2&59.3&&85.8&93.4&96.6\\
            \scriptsize &&&3&1.0&5.1&11.4&&14.1&32.1&45.2&&76.1&92.0&95.4 &&100.0&100.0&100.0&&0.7&5.3&10.5&&6.0&16.8&26.7&&25.7&47.6&59.2&&84.5&94.2&97.3\\
            \scriptsize &&&5&1.1&5.4&12.0&& 15.5&32.2&45.4&&76.2&92.2&95.1 &&100.0&100.0&100.0&&0.8&5.7&10.9&&6.4&16.5&25.5&&24.9&47.8&59.8&&83.8&94.4&96.6\\
            \scriptsize &&&6&1.1&5.6&11.3&&14.1&32.2&45.4&&76.2&91.2 &95.8 &&100.0&100.0&100.0&&0.9&6.2&11.1&&6.0&16.8&24.9&&24.8&48.3&60.3&&83.4&93.6&96.6\\
            \scriptsize &&&12&1.2&5.2&11.1&&13.6&33.1&46.3&&75.7&92.1&95.6 &&100.0&100.0&100.0&&0.7&5.7&10.8&&6.4&16.1&26.1&&24.1&46.0&59.6&&81.9&94.1&97.2\\
            \scriptsize &&&15&1.6&5.9&10.9&&15.7&33.0&46.2&&75.0&91.5&95.7 &&100.0&100.0&100.0&&1.2&5.8&10.9&&6.3&17.3&25.4&&24.3&46.5&60.7&&82.6&94.8&96.7\\
            \cline{4-19} \cline{21-35}
            \scriptsize &&\multirow{5}{*}{\text{Exp(1)}}&1&1.0 & 4.4&  9.6&&13.9 &29.5& 41.1&&73.4& 88.6& 92.6&&99.5& 100.0& 100.0  &&0.0&4.4&11.1&&2.1&11.9&23.6&&11.7&43.6&63.2&&84.9&95.3&98.0\\
            \scriptsize &&&3&0.8&4.2&9.4&&12.8& 29.2& 41.1&&73.9& 90.3& 92.6&&99.1& 99.6& 99.9&&0.0&4.0&10.9&&1.4&13.6&27.2&&11.6&43.9&62.1&& 84.6& 94.3& 97.4\\
            \scriptsize &&&5&0.7&4.0&9.7&& 14.1& 29.3& 41.3&&74.0& 89.5& 92.3&&99.2& 99.6& 99.8&&0.2&4.2&10.2&&1.6&13.4&25.6&&11.3&42.3&60.8&&83.8& 94.4& 96.6\\
            \scriptsize &&&6&0.5&3.6&9.5&&12.8& 29.3& 41.3&&74.0& 88.5& 93.0&&99.3& 99.4& 100.0&&0.0&4.2&11.1&&1.8&14.4&26.2&&9.5&39.7& 59.0 &&83.4& 93.7& 96.6\\
            \scriptsize&&&12&0.4&3.4&9.6&&12.4& 30.1& 42.1&& 73.5& 89.4& 92.8&&99.3& 99.5& 99.9&&0.0&3.3&11.1&&0.8&12.6&26.6&&7.9&40.0&59.8&&81.9&94.1&97.2\\
            \scriptsize &&&15&0.3&3.3&9.3&&14.3& 30.0& 42.0&&72.8& 88.8&92.9&&99.2& 99.3& 99.7&&0.0&3.1&11.1&&0.6&12.6&25.2&&9.8& 41.5& 59.5&&82.6& 94.9& 96.8\\
            \cline{2-19} \cline{21-35}
            \scriptsize &$\text{T}_{n}$&\multirow{3}{*}{\text{Daniell}}&6&0.4&4.0&7.7&&6.4&7.2& 16.9&&49.6&51.3&77.9&&100.0&100.0&100.0 &&6.3&11.2&17.2&&1.5&5.5&12.0&&6.4&21.6&38.2&&83.8&95.8&98.2\\
            \scriptsize &&&13&0.2&4.1&11.2&&3.7&6.4&15.3&&39.7&41.1&62.2&&100.0&100.0&100.0&&4.4&11.4&18.6&&2.2&6.8&11.4&&8.8&24.0&32.0&&84.8&94.9&96.4\\
            \scriptsize &&&23&0.3&5.1&11.9&&2.8&6.9&14.2&&27.2&29.7&49.9&&99.9&99.9&100.0&&3.1&10.3&17.8&&1.8&5.6&10.9&&8.2&19.7&27.9&&78.8&88.1&93.6\\
            \cline{4-19} \cline{21-35}
            \scriptsize & &\multirow{3}{*}{\text{Parzen}}&6&0.4&1.7&7.3&&6.4&12.0&16.9&&59.6&71.9& 78.2&&100.0&100.0&100.0  &&11.2&16.7&23.0&&2.5&9.5&20.4&&15.8&35.8&53.6&&94.7&98.3&99.5\\
            \scriptsize &&&13&0.2&4.2& 11.2 &&3.7&9.3& 15.7&&39.7& 55.0& 62.1&&100.0 &100.0&100.0 &&5.7&12.9&21.3&&2.1&8.4&16.6&&14.1&31.4&44.8&& 91.4& 97.5& 98.7\\
            \scriptsize &&&23&0.3& 5.2 &12.4 &&2.8&8.6& 14.6&&27.2& 42.3& 50.3&& 99.9&100.0&100.0&&4.8&12.2&20.7&&2.2&8.4&13.3&&12.7&27.4&38.6&& 87.2& 95.0& 97.9\\
            \hline
        \end{tabular}
    }
\end{sidewaystable}

\subsection{长表格}

如某个表需要转页接排,可以使用 longtable 宏包,需要在随后的各页上重复表的编号。
编号后跟表题(可省略)和“(续)”,置于表上方。续表均应重复表头。如表 \ref{tab:longTable} 所示。

当一个张表内容过多时,建议将该表置于附录中,如附录 \ref{tab:appendix-table} 所示。

\begin{longtable}{cccccccc}
    \caption{跨页长表格} \\ % 换页前标题
    \toprule[1.5pt]
        \textbf{表头 1} & \textbf{表头 2} & \textbf{表头 3} & \textbf{表头 4} & \textbf{表头 5} & \textbf{表头 6} & \textbf{表头 7} & \textbf{表头 8} \\ % 换页前表头
    \midrule[0.8pt]
    \endfirsthead
    
    \caption[]{跨页长表格(续)} \\ % 换页后标题
    \toprule[1.5pt]
        \textbf{表头 1} & \textbf{表头 2} & \textbf{表头 3} & \textbf{表头 4} & \textbf{表头 5} & \textbf{表头 6} & \textbf{表头 7} & \textbf{表头 8} \\  % 换页后表头
    \midrule[0.8pt]
    \endhead
        \bottomrule[1.5pt]
        % \multicolumn{7}{r}{\textit{\zihao{5} \songti 接下页}} \\ 
    \endfoot 
    \bottomrule[1.5pt]
    \endlastfoot
    \label{tab:longTable}
        Row 01 & 01-01 & 01-02 & 01-03 & 01-04 & 01-05 & 01-06 & 01-07 \\
        Row 02 & 02-01 & 02-02 & 02-03 & 02-04 & 02-05 & 02-06 & 02-07 \\
        Row 03 & 03-01 & 03-02 & 03-03 & 03-04 & 03-05 & 03-06 & 03-07 \\
        Row 04 & 04-01 & 04-02 & 04-03 & 04-04 & 04-05 & 04-06 & 04-07 \\
        Row 05 & 05-01 & 05-02 & 05-03 & 05-04 & 05-05 & 05-06 & 05-07 \\
        Row 06 & 06-01 & 06-02 & 06-03 & 06-04 & 06-05 & 06-06 & 06-07 \\
        Row 07 & 07-01 & 07-02 & 07-03 & 07-04 & 07-05 & 07-06 & 07-07 \\
        Row 08 & 08-01 & 08-02 & 08-03 & 08-04 & 08-05 & 08-06 & 08-07 \\
        Row 09 & 09-01 & 09-02 & 09-03 & 09-04 & 09-05 & 09-06 & 09-07 \\
        Row 10 & 10-01 & 10-02 & 10-03 & 10-04 & 10-05 & 10-06 & 10-07 \\
        Row 11 & 11-01 & 11-02 & 11-03 & 11-04 & 11-05 & 11-06 & 11-07 \\
        Row 12 & 12-01 & 12-02 & 12-03 & 12-04 & 12-05 & 12-06 & 12-07 \\
        Row 13 & 13-01 & 13-02 & 13-03 & 13-04 & 13-05 & 13-06 & 13-07 \\
        Row 14 & 14-01 & 14-02 & 14-03 & 14-04 & 14-05 & 14-06 & 14-07 \\
        Row 15 & 15-01 & 15-02 & 15-03 & 15-04 & 15-05 & 15-06 & 15-07 \\
        Row 16 & 16-01 & 16-02 & 16-03 & 16-04 & 16-05 & 16-06 & 16-07 \\
        Row 17 & 17-01 & 17-02 & 17-03 & 17-04 & 17-05 & 17-06 & 17-07 \\
        Row 18 & 18-01 & 18-02 & 18-03 & 18-04 & 18-05 & 18-06 & 18-07 \\
        Row 19 & 19-01 & 19-02 & 19-03 & 19-04 & 19-05 & 19-06 & 19-07 \\
        Row 20 & 20-01 & 20-02 & 20-03 & 20-04 & 20-05 & 20-06 & 20-07 \\
        Row 21 & 21-01 & 21-02 & 21-03 & 21-04 & 21-05 & 21-06 & 21-07 \\
        Row 22 & 22-01 & 22-02 & 22-03 & 22-04 & 22-05 & 22-06 & 22-07 \\
        Row 23 & 23-01 & 23-02 & 23-03 & 23-04 & 23-05 & 23-06 & 23-07 \\
        Row 24 & 24-01 & 24-02 & 24-03 & 24-04 & 24-05 & 24-06 & 24-07 \\
        Row 25 & 25-01 & 25-02 & 25-03 & 25-04 & 25-05 & 25-06 & 25-07 \\
        Row 26 & 26-01 & 26-02 & 26-03 & 26-04 & 26-05 & 26-06 & 26-07 \\
        Row 27 & 27-01 & 27-02 & 27-03 & 27-04 & 27-05 & 27-06 & 27-07 \\
        Row 28 & 28-01 & 28-02 & 28-03 & 28-04 & 28-05 & 28-06 & 28-07 \\
        Row 29 & 29-01 & 29-02 & 29-03 & 29-04 & 29-05 & 29-06 & 29-07 \\
        Row 30 & 30-01 & 30-02 & 30-03 & 30-04 & 30-05 & 30-06 & 30-07 \\
        Row 31 & 31-01 & 31-02 & 31-03 & 31-04 & 31-05 & 31-06 & 31-07 \\
        Row 32 & 32-01 & 32-02 & 32-03 & 32-04 & 32-05 & 32-06 & 32-07 \\
        Row 33 & 33-01 & 33-02 & 33-03 & 33-04 & 33-05 & 33-06 & 33-07 \\
        Row 34 & 34-01 & 34-02 & 34-03 & 34-04 & 34-05 & 34-06 & 34-07 \\
        Row 35 & 35-01 & 35-02 & 35-03 & 35-04 & 35-05 & 35-06 & 35-07 \\
        Row 36 & 36-01 & 36-02 & 36-03 & 36-04 & 36-05 & 36-06 & 36-07 \\
        Row 37 & 37-01 & 37-02 & 37-03 & 37-04 & 37-05 & 37-06 & 37-07 \\
        Row 38 & 38-01 & 38-02 & 38-03 & 38-04 & 38-05 & 38-06 & 38-07 \\
        Row 39 & 39-01 & 39-02 & 39-03 & 39-04 & 39-05 & 39-06 & 39-07 \\
        Row 40 & 40-01 & 40-02 & 40-03 & 40-04 & 40-05 & 40-06 & 40-07 \\
        Row 41 & 41-01 & 41-02 & 41-03 & 41-04 & 41-05 & 41-06 & 41-07 \\
        Row 42 & 42-01 & 42-02 & 42-03 & 42-04 & 42-05 & 42-06 & 42-07 \\
        Row 43 & 43-01 & 43-02 & 43-03 & 43-04 & 43-05 & 43-06 & 43-07 \\
        Row 44 & 44-01 & 44-02 & 44-03 & 44-04 & 44-05 & 44-06 & 44-07 \\
\end{longtable}

\chapter{引用文献相关}

模板使用 \hologo{BibTeX} 处理引用参考文献。本章主要介绍 \hologo{BibTeX} 配合 \pkg{natbib} 宏包的主要使用方法\cite{zhangkun1994,zhukezhen1973,dupont1974bone,zhengkaiqing1987,jiangxizhou1980,jianduju1994,merkt1995rotational,mellinger1996laser,bixon1996dynamics,mahui1995,carlson1981two,taylor1983scanning,taylor1981study,shimizu1983laser,atkinson1982experimental,kusch1975perturbations,guangxi1993,huosini1989guwu,wangfuzhi1865songlun,zhaoyaodong1998xinshidai,biaozhunhua2002tushu,chubanzhuanye2004,who1970factors,peebles2001probability,baishunong1998zhiwu,weinstein1974pathogenic,hanjiren1985lun,dizhi1936dizhi,tushuguan1957tushuguanxue,aaas1883science,fugang2000fengsha,xiaoyu2001chubanye,oclc2000about,scitor2000project}。

\section{顺序编码制}

在顺序编码制下,默认的 \cs{cite\{\}} 命令同 \cs{citep\{\}} 一样,序号置于方括号中,引文页码会放在括号外。同一处引用的连续序号会自动用短横线连接。

\begin{table}[H]
  \centering
  \caption{顺序编码制中的对应关系}
      \begin{tabular}{l@{\quad$\Rightarrow$\quad}l}
      \verb|\cite{zhangkun1994}|               & \cite{zhangkun1994}               \\
      \verb|\citet{zhangkun1994}|              & \citet{zhangkun1994}              \\
      \verb|\citep{zhangkun1994}|              & \citep{zhangkun1994}              \\
      \verb|\cite[42]{zhangkun1994}|           & \cite[42]{zhangkun1994}           \\
      \verb|\cite{zhangkun1994,zhukezhen1973}| & \cite{zhangkun1994,zhukezhen1973} \\
      \end{tabular}
\end{table}

也可以通过 \cs{setcitestyle\{numbers\}} 设置引用样式为取消上标,将数字序号作为文字的一部分。

\setcitestyle{numbers} % 修改引用样式为取消上标格式

\begin{table}[H]
  \centering
  \caption{取消上标格式的顺序编码制中的对应关系}
      \begin{tabular}{l@{\quad$\Rightarrow$\quad}l}
      \verb|\cite{zhangkun1994}|               & \cite{zhangkun1994}               \\
      \verb|\citet{zhangkun1994}|              & \citet{zhangkun1994}              \\
      \verb|\citep{zhangkun1994}|              & \citep{zhangkun1994}              \\
      \verb|\cite[42]{zhangkun1994}|           & \cite[42]{zhangkun1994}           \\
      \verb|\cite{zhangkun1994,zhukezhen1973}| & \cite{zhangkun1994,zhukezhen1973} \\
      \end{tabular}
\end{table}

\section{著者-出版年制}

可以通过 \cs{setcitestyle\{authoryear\}} 设置属性引用样式为著者-出版年制,其中 \cs{cite\{\}} 
跟 \cs{citet\{\}} 效果一样。

\setcitestyle{authoryear} % 修改引用样式为著者-出版年制

\begin{table}[H]
  \centering
  \caption{著者-出版年制中的对应关系}
      \begin{tabular}{l@{\quad$\Rightarrow$\quad}l}
      \verb|\cite{zhangkun1994}|           & \cite{zhangkun1994}           \\
      \verb|\citet{zhangkun1994}|          & \citet{zhangkun1994}          \\
      \verb|\citep{zhangkun1994}|          & \citep{zhangkun1994}          \\
      \verb|\cite[42]{zhangkun1994}|       & \cite[42]{zhangkun1994}       \\
      \verb|\cite{zhangkun1994,mahui1995}| & \cite{zhangkun1994,mahui1995} \\
      \end{tabular}
\end{table}

\setcitestyle{super} % 修改引用样式为默认

\chapter{算法及代码相关}

\section{算法}

算法(伪代码)环境可以使用 \pkg{algorithms} 宏包。算法 \ref{alg1} 为演示的示例。

\begin{algorithm}
	\caption{STVMD based on STFT}
	\label{alg1}
	\begin{algorithmic}[1]
		\STATE Initialization:$\left\{ {s_{k,t}^1} \right\},\left\{ {\omega _{k,t}^1} \right\},\lambda _t^1,n \leftarrow 0$
		\STATE  ${s_{r,t}}\left( \omega  \right) = \int_0^{ + \infty } {{s_r}\left( \tau  \right){w_h}\left( {t - \tau } \right)} \exp \left( {j\omega \tau } \right)d\tau $   (via STFT)
		\REPEAT
		\STATE $n \leftarrow n + 1$
		\STATE Update $ s_{k,t}^{n + 1} $ based on Equation \eqref{eq:example1}
		\STATE Update $\omega _{k,t}^{n + 1}$ based on Equation \eqref{eq:example1}
		\STATE Update $\lambda _t^{n + 1} $ based on Equation \eqref{eq:example1}
		\UNTIL $\sum\limits_{k=1}^P  {{{\left\| {s_{k,t}^{n + 1}\left( \omega  \right) - s_{k,t}^n\left( \omega  \right)} \right\|_2^2} \mathord{\left/
					{\vphantom {{\left\| {s_{k,t}^{n + 1}\left( \omega  \right) - s_{k,t}^n\left( \omega  \right)} \right\|_2^2} {\left\| {s_{k,t}^n\left( \omega  \right)} \right\|_2^2}}} \right.
					\kern-\nulldelimiterspace} {\left\| {s_{k,t}^n\left( \omega  \right)} \right\|_2^2}}}  < \varepsilon $  
		\STATE   Update ${s_k}\left( t \right)$ based on Equation \eqref{eq:example1}  (via ISTFT)
		\ENSURE  decomposed modes $ \left\{ {{s_k}\left( t \right)} \right\}$, $\left\{ {{\omega _k}\left( t \right)} \right\}$
	\end{algorithmic}  
\end{algorithm}

\section{代码}

代码环境可以使用 \pkg{lstlisting} 宏包。宏包可以自定义代码中关键字的高亮、代码块的边框、行号的样式等。

代码 \ref{code:Python} 是一段 Pyhton 代码示例,代码 \ref{code:Java} 是一段 Java 代码示例。

\begin{lstlisting}[label=code:Python, language=Python, caption=Python代码测试]
def countingSort(arr):
    if len(arr) == 0:
        return arr

    # 获取数组中最大值和最小值
    minValue = min(arr)
    maxValue = max(arr)

    # 初始化计数数组
    countArr = [0] * (maxValue - minValue + 1)

    # 计算每个数字出现的次数
    for i in range(len(arr)):
        countArr[arr[i] - minValue] += 1

    # 计算每个数字在排序后数组中的位置
    for i in range(1, len(countArr)):
        countArr[i] += countArr[i-1]

    # 排序数组
    sortedArr = [0] * len(arr)
    for i in range(len(arr)):
        index = countArr[arr[i] - minValue] - 1
        sortedArr[index] = arr[i]
        countArr[arr[i] - minValue] -= 1

    return sortedArr
\end{lstlisting}

\begin{lstlisting}[label=code:Java, language=Java, caption=Java代码测试]
/**
 * 归并排序
 *
 * @param array
 * @return
 */
public static int[] MergeSort(int[] array) {
    if (array.length < 2) return array;
    int mid = array.length / 2;
    
    int[] left = Arrays.copyOfRange(array, 0, mid);
    int[] right = Arrays.copyOfRange(array, mid, array.length);
    return merge(MergeSort(left), MergeSort(right));
}

/**
 * 归并排序——将两段排序好的数组结合成一个排序数组
 *
 * @param left
 * @param right
 * @return
 */
public static int[] merge(int[] left, int[] right) {
    int[] result = new int[left.length + right.length];
    
    for (int index = 0, i = 0, j = 0; index < result.length; index++) {
        if (i >= left.length)
            result[index] = right[j++];
        else if (j >= right.length)
            result[index] = left[i++];
        else if (left[i] > right[j])
            result[index] = right[j++];
        else
            result[index] = left[i++];
    }
    return result;
}
\end{lstlisting}

\chapter{数学相关}

\section{数学符号}

中文论文的数学符号默认遵循 GB/T 3102.11—1993《物理科学和技术中使用的数学符号》\footnote{原 GB 3102.11—1993,自 2017 年 3 月 23 日起,该标准转为推荐性标准。}。该标准参照采纳 ISO 31-11:1992 \footnote{目前已更新为 ISO 80000-2:2019。}。

英文论文的数学符号使用 \TeX{} 默认的样式。

关于量和单位推荐使用 \href{http://mirrors.ctan.org/macros/latex/contrib/siunitx/siunitx.pdf}{\pkg{siunitx}} 宏包,可以方便地处理希腊字母以及数字与单位之间的空白,比如:
\SI{6.4e6}{m},
\SI{9}{\micro\meter},
\SI{30}{kg.m.s^{-1}},
\SI{20}{\degreeCelsius}。

表 \ref{tab:number} 展示了一些数字和单位的正确写法以及常见的错误写法。

\begin{table}[H]
    \centering
    \caption{数字与单位示范}
    \label{tab:number}
    \renewcommand\arraystretch{1.5} % 定义表格行距
    \setlength{\tabcolsep}{20pt} % 定义列间宽度
    \begin{tabular}{@{}cc@{}}
        \toprule[1.5pt]
        正确示例 & 错误示例 \\ 
        \midrule[0.8pt]
        \num{12345,67890} & 12345.67890 \\
        \num{.3e45} & 0.3 $\times$ 10\textsuperscript{45} \\
        \si{\kilo\gram\metre\per\square\second} & kg m s\textsuperscript{-2} \\
        \si{\square\volt\cubic\lumen\per\farad} & $V^{2}lm^{3}F^{-1}$ \\
        \SI[mode=text]{1.23}{J.mol^{-1}.K^{-1}} & 1.23J mol\textsuperscript{-1}K\textsuperscript{-1} \\
        \SI[per-mode=symbol]{1.99}[\$]{\per\kilogram} & \$ 1.99/kg \\
        \SI[per-mode=fraction]{1,345}{\coulomb\per\mole} & 1.345$\frac{C}{mol}$ \\ 
        \bottomrule[1.5pt]
    \end{tabular}
\end{table}

\section{数学公式}

数学公式可以使用 \env{equation} 和 \env{equation*} 环境。注意数学公式的引用应前后带括号,建议使用 \cs{eqref} 命令,如公式 \eqref{eq:example1}。

\begin{equation}
    \oint_{\partial \Sigma} \mathbf{E} \cdot d\mathbf{l} = -\frac{d}{dt} \int_{\Sigma} \mathbf{B} \cdot d\mathbf{S}
    \label{eq:example1}
\end{equation}

如果需要多行公式尽可能在等号处对齐,可以使用 \env{align} 环境。
\begin{align}
    a & = b + c + d + e \\
    b & = f + g + h + i + j + k + l + m \\
    c & = o + p + q + r + s + t
\end{align}

如果不需要按等号对齐, 只需罗列数个公式, 可以使用 \env{gather} 环境。

\begin{gather}
    a = b + c \\
    d = e + f + g \\
    h + i = j + k \\
    l + m = n
\end{gather}

如果需要多个公式组在一起共用一个编号, 编号位于公式的居中位置,推荐使用 \env{aligned} 环境。使用效果如公式 \eqref{eq:example2} 所示。

\begin{equation}
    \label{eq:example2}
    \left\{
        \begin{aligned}
          &-\frac{\mathrm{d}^{2} u}{\mathrm{d} x^{2}}+\frac{\mathrm{d} u}{\mathrm{d} x}=\pi^{2} \sin (\pi x)+\pi \cos (\pi x),\quad x \in [0,1], \\
          &u(0)=0,\quad u(1)=0.
        \end{aligned} 
    \right.
\end{equation}

\section{数学定理}

定理环境的格式可以使用 \pkg{amsthm} 或者 \pkg{ntheorem} 宏包配置。一些使用的示例如下文所示。

\begin{theorem}[Lindeberg--Lévy 中心极限定理]
    设随机变量 $X_1, X_2, \dots, X_n$ 独立同分布, 且具有期望 $\mu$ 和有限的方差 $\sigma^2 \ne 0$,
    记 $\bar{X}_n = \frac{1}{n} \sum_{i+1}^n X_i$,则
    \begin{equation}
        \lim_{n \to \infty} P \left(\frac{\sqrt{n} \left( \bar{X}_n - \mu \right)}{\sigma} \le z \right) = \Phi(z),
    \end{equation}
    其中 $\Phi(z)$ 是标准正态分布的分布函数。
\end{theorem}

\begin{lemma}\label{lemma-convergence} 
    (参考文献\cite{aaas1883science})假设单步法具有 $p$ 阶精度, 増量函数 $\varphi(x_{n}, u_{n}, h)$ 关于 $u$ 满足 {\rm Lipschitz} 条件
    \begin{equation}\label{eqn:3}
        |\varphi(x, u, h)-\varphi(x, \bar{u}, h)| \leqslant L_{\varphi}|u-\bar{u}|
    \end{equation}
\end{lemma}

\begin{corollary}\label{col-convergence}
    假设单步法具有 $p$ 阶精度, 且増量函数 $\varphi(x_{n}, u_{n}, h)$ 关于 $u$ 满足 {\rm Lipschitz} 条件。由证明 \ref{proof1}、公理 \ref{axiom1}、性质 \ref{property1}、假设 \ref{assumption1}、注 \ref{remark1}、例 \ref{example1} 可得。
    \begin{equation}\label{eqn:5}
        |\varphi(x, u, h)-\varphi(x, \bar{u}, h)| \leqslant L_{\varphi}|u-\bar{u}|
    \end{equation}
\end{corollary}

\begin{proof}\label{proof1}
    Trivial. \QED
\end{proof}

\begin{axiom}\label{axiom1}
    Axiom.
\end{axiom}

\begin{property}\label{property1}
    Property. 
\end{property}

\begin{assumption}\label{assumption1}
    Assumption.
\end{assumption}

\begin{remark}\label{remark1}
    Remark.
\end{remark}

\begin{example}\label{example1}
    Example.
\end{example}

%%%%%%%%%%%%%%%%%%%%%%%%  参考文献  %%%%%%%%%%%%%%%%%%%%%%%%

\begin{references}
    \bibliography{references.bib} % 指定.bib文件路径
\end{references}

%%%%%%%%%%%%%%%%%%%%%%%%%  附录  %%%%%%%%%%%%%%%%%%%%%%%%%%

\StartAppendix % 启用附录

\chapter{计算机常见术语}
\label{tab:appendix-table}

\begin{center}
    \begin{longtable}{|c|c|}
        \hline
        \textbf{英文} & \textbf{中文} \\
        \hline
    \endfirsthead
        \hline
        \textbf{英文} & \textbf{中文} \\
        \hline
    \endhead
        \hline
    \endfoot 
    \endlastfoot
        Abstract & 摘要;抽象的 \\ 
        Abstraction & 抽象 \\ 
        Access & 访问 \\ 
        Accessibility & 无障碍;辅助功能 (win/mac) \\ 
        Activate, Activation & 激活 \\ 
        Active & 使用中的;现用的;有效的;激活的 \\ 
        Adapter, Adaptor & 适配卡,适配器 \\ 
        Add & 添加 \\ 
        Address & 位址,地址 \\ 
        Advanced & 高级的 \\ 
        Aggregation & 聚合 \\ 
        AI (Artificial intelligence) & 人工智能 \\ 
        Algorithm & 算法 \\ 
        Allocate & 分配 \\ 
        Allocator & 分配器 \\ 
        Annotation & 注释 (win);注解 (mac) \\ 
        App bundle & 应用程序包 (win);App 捆绑包 (mac) \\ 
        Application & 应用;应用程序 \\ 
        Apply & 应用 \\ 
        Architecture & 架构;结构 \\ 
        Argument & 参数(也称为实际参数,实参) \\ 
        Arity & 参数数量 \\ 
        Artifact & 项目 (win);成品 (mac) \\ 
        Array & 数组 \\ 
        Assembly language & 汇编语言 \\ 
        Assert, Assertion & 断言 (win);声明 (win/mac);论断 (mac) \\ 
        Assign, Assignment & 分配;(编程)赋值 \\ 
        Assignment operator & 赋值运算符 \\ 
        Asynchronize & 异步 \\ 
        Asynchronous & 异步的 \\ 
        Atomic & 原子的 \\ 
        Attribute & 属性 \\ 
        Authenticate, Authentication & 验证,认证 \\ 
        Authorize, Authorization & 授权 \\ 
        \hline
    \end{longtable}
    
\end{center}

\chapter{符号与缩略语}

\begin{denotation}
  \item[PI] 聚酰亚胺
  \item[MPI] 聚酰亚胺模型化合物,N-苯基邻苯酰亚胺
  \item[PBI] 聚苯并咪唑
  \item[MPBI] 聚苯并咪唑模型化合物,N-苯基苯并咪唑
  \item[PY] 聚吡咙
  \item[PMDA-BDA] 均苯四酸二酐与联苯四胺合成的聚吡咙薄膜
  \item[MPY] 聚吡咙模型化合物
  \item[As-PPT] 聚苯基不对称三嗪
  \item[MAsPPT] 聚苯基不对称三嗪单模型化合物,3,5,6-三苯基-1,2,4-三嗪
  \item[DMAsPPT] 聚苯基不对称三嗪双模型化合物(水解实验模型化合物)
  \item[S-PPT] 聚苯基对称三嗪
  \item[MSPPT] 聚苯基对称三嗪模型化合物,2,4,6-三苯基-1,3,5-三嗪
  \item[PPQ] 聚苯基喹噁啉
  \item[MPPQ] 聚苯基喹噁啉模型化合物,3,4-二苯基苯并二嗪
  \item[HMPI] 聚酰亚胺模型化合物的质子化产物
  \item[HMPY] 聚吡咙模型化合物的质子化产物
  \item[HMPBI] 聚苯并咪唑模型化合物的质子化产物
  \item[HMAsPPT] 聚苯基不对称三嗪模型化合物的质子化产物
  \item[HMSPPT] 聚苯基对称三嗪模型化合物的质子化产物
  \item[HMPPQ] 聚苯基喹噁啉模型化合物的质子化产物
  \item[PDT] 热分解温度
  \item[HPLC] 高效液相色谱(High Performance Liquid Chromatography)
  \item[HPCE] 高效毛细管电泳色谱(High Performance Capillary lectrophoresis)
  \item[LC-MS] 液相色谱-质谱联用(Liquid chromatography-Mass Spectrum)
  \item[TIC] 总离子浓度(Total Ion Content)
  \item[\textit{ab initio}] 基于第一原理的量子化学计算方法,常称从头算法
  \item[DFT] 密度泛函理论(Density Functional Theory)
  \item[$E_a$] 化学反应的活化能(Activation Energy)
  \item[ZPE] 零点振动能(Zero Vibration Energy)
  \item[PES] 势能面(Potential Energy Surface)
  \item[TS] 过渡态(Transition State)
  \item[TST] 过渡态理论(Transition State Theory)
  \item[$\upDelta G^\neq$] 活化自由能(Activation Free Energy)
  \item[$\kappa$] 传输系数(Transmission Coefficient)
  \item[IRC] 内禀反应坐标(Intrinsic Reaction Coordinates)
  \item[$\nu_i$] 虚频(Imaginary Frequency)
  \item[ONIOM] 分层算法(Our own N-layered Integrated molecular Orbital and molecular Mechanics)
\end{denotation}

\chapter{数学符号测试}

\[ x = \frac{-b \pm \sqrt{b^2 - 4ac}}{2a} \]

\[ \binom{n}{k} = \frac{n!}{k!(n-k)!} \]

\[ [a,b) = \{ x\in\mathbb{R} \mid a \le x < b \} \]

\[ m = \frac{\Delta y}{\Delta x} = \frac{y_2 - y_1}{x_2 - x_1} \]

\[ \int_a^b f(x)\ dx = \lim_{n\to\infty} \left( \sum_{i=1}^n f(x_i^*) \Delta x_i \right) \]

\[ \frac{d}{dx} \left[ \int_a^x f'(t)\ dt \right] = f(x) \]

\[ \sin^2\alpha + \cos^2\alpha = 1 \]

\begin{center}
    \begin{tabular}{ll}
    $\leq$ &  $\backslash$leq \\
    $\geq$ &  $\backslash$geq \\
    $\neq$ &  $\backslash$neq \\
    $\nleq$ &  $\backslash$nleq \\
    $\ngeq$ &  $\backslash$ngeq \\
    $\cong$ &  $\backslash$cong \\
    $\equiv$ &  $\backslash$equiv \\
    $\sim$ &  $\backslash$sim \\
    $\approx$ &  $\backslash$approx \\
    $\doteqdot$ &  $\backslash$doteqdot \\
    $\times$ &  $\backslash$times \\
    $\cdot $ &  $\backslash$cdot \\
    $\ast $ &  $\backslash$ast \\
    $\div$ &  $\backslash$div \\
    $\pm$ &  $\backslash$pm \\
    $\mp$ &  $\backslash$mp \\
    $\bigcirc$ &  $\backslash$bigcirc \\
    $\oplus$ &  $\backslash$oplus \\
    $\otimes$ &  $\backslash$otimes \\
    \end{tabular}
    \hspace*{1ex}
    \begin{tabular}{ll}
    $\propto $ &  $\backslash$propto \\
    $\cdots $ &  $\backslash$cdots \\
    $\dots $ &  $\backslash$dots \\
    $\because$ &  $\backslash$because \\
    $\therefore$ &  $\backslash$therefore \\
    $\forall$ &  $\backslash$forall \\
    $\exists$ &  $\backslash$exists \\
    $\in$ &  $\backslash$in \\
    $\subset $ &  $\backslash$subset \\
    $\subseteq $ &  $\backslash$subseteq \\
    $\varnothing $ &  $\backslash$varnothing  \\
    $\cap $ &  $\backslash$cap \\
    $\cup $ &  $\backslash$cup \\
    $\setminus $ &  $\backslash$setminus \\
    $\wedge $ &  $\backslash$wedge \\
    $\vee $ &  $\backslash$vee \\
    $\Rightarrow$ &  $\backslash$Rightarrow \\
    $\rightarrow$ &  $\backslash$rightarrow \\
    $\mapsto$ &  $\backslash$mapsto \\
    \end{tabular}
    \hspace*{1ex}
    \begin{tabular}{ll}
    \$ & $\backslash$\$ \\
    \& & $\backslash$\& \\
    \% & $\backslash$\% \\
    $\backslash$ & $\backslash$backslash \\
    $\sharp$ & $\backslash$sharp \\
    $\partial$ &  $\backslash$partial \\
    $90^\circ$ &  90$^\wedge\backslash$circ \\
    $\parallel$ &  $\backslash$parallel \\
    $\bot$ &  $\backslash$bot \\
    $\triangle$ &  $\backslash$triangle \\
    $\nabla$ &   $\backslash$nabla \\
    $\square$ &  $\backslash$square \\
    $\angle$ &  $\backslash$angle \\
    $\Pi$ &  $\backslash$Pi \\
    $\Theta$ &  $\backslash$Theta \\
    $\Gamma$ &  $\backslash$Gamma \\
    $\Delta$ &  $\backslash$Delta \\
    $\Omega$ &  $\backslash$Omega \\
    $\Sigma$ &  $\backslash$Sigma \\
    \end{tabular}
    \hspace*{1ex}
    \begin{tabular}{ll}
    $\alpha$ &  $\backslash$alpha \\
    $\beta$ &  $\backslash$beta \\
    $\epsilon$ &  $\backslash$epsilon \\
    $\zeta$ &  $\backslash$zeta \\
    $\eta$ &  $\backslash$eta \\
    $\kappa$ &  $\backslash$kappa \\
    $\lambda$ &  $\backslash$lambda \\
    $\mu$ &  $\backslash$mu \\
    $\xi$ &  $\backslash$xi \\
    $\rho$ &  $\backslash$rho \\
    $\tau$ &  $\backslash$tau \\
    $\phi$ &  $\backslash$phi \\
    $\psi$ &  $\backslash$psi \\
    $\pi$ &  $\backslash$pi \\
    $\theta$ &  $\backslash$theta \\
    $\gamma$ &  $\backslash$gamma\\
    $\delta$ &  $\backslash$delta \\
    $\omega$ &  $\backslash$omega \\
    $\sigma$ &  $\backslash$sigma \\
    \end{tabular}
\end{center}

%%%%%%%%%%%%%%%%%%%%%%%  正文后页眉页脚  %%%%%%%%%%%%%%%%%%%%%%

% 页眉(关闭页眉务必将页眉类型设为empty)
\Header
    {common} % 页眉类型:common、publish、empty
    {1pt} % 上分隔线宽度
    {1pt} % 两线距离
    {0.5pt} % 下分割线宽度
    {} % 页眉左自定义内容(文本或图片)
    {\includegraphics[width=0.185\textwidth]{DGUT-title-CN.pdf}} % 页眉中自定义内容(文本或图片)
    {} % 页眉右自定义内容(文本或图片)

%============================================%

% 页脚(关闭页脚务必将页脚类型设为empty)
\Footer
    {common} % 页脚类型:common、publish、empty
    {0pt} % 上分隔线宽度
    {0pt} % 两线距离
    {0pt} % 下分割线宽度
    {} % 页脚左自定义内容(文本或图片)
    {\thepage} % 页脚中自定义内容(文本或图片)
    {} % 页脚右自定义内容(文本或图片)

%============================================%

% 页数样式 参数:#1 起始页数
% \setRomanPageNumber{1} % 设置罗马数字页码
% \setArabicPageNumber{1} % 设置阿拉伯数字页码

%%%%%%%%%%%%%%%%%%%%%%%%%  致谢  %%%%%%%%%%%%%%%%%%%%%%%%%

\StartAcknowledgements % 启用致谢

感谢各参考模板的帮助。感谢你的使用。

\end{document}